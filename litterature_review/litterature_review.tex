\documentclass{article}
\title{Review of the litterature}
\author
{
    Guillermo Martinez (matricule x)
    \and
    Dereck Piché (matricule 20177385)
    \and
    Jonas Gabirot (matricule 20185863)
}


\begin{document}
\maketitle

\section*{Aptamers}
\subsection*{Description of aptamers}
*insert*
\subsection*{Motivation behind the study of Aptamers}


\section*{Machine learning algorithms}
\subsection*{Multilayer Perceptrons}
This is the deep architecture that started it all 
and the one we shall use to create our baseline since it does 
not posess any bias towards the positionning of 
the different tokens. The use of MLP's is made possible by the fact 
that we are only dealing with 4 different possible elements as our tokens
(A,C,G,T).

\subsection*{Reccurent Neural Networks}
The broad definition of a recurrent neural network 
is that it is the subclass of neural networks \cite{transformers} wich have cycles in the layers. 
RNN's were the default deep learning models used for sequences
of tokens for a while, until the recent rise of transformers (especially LTSM RNN's) 
These RNN's processed multiple tokens one at a time by storing information in the so called
hidden state of the network. Since you have to process each token one at 
a time, their is no parallelization. 
One advantage of recurrent neural network is that they grow 
linearly with respect to the amount of tokens as input, which means 
that their really isn't a practical limit to the number of tokens that
it can process with reasonable time complexity. 
\subsection*{Transformers}
According to our assumptions, the transformer 
architecture \cite{transformers} is by far the most appropriate
for our task. Transformers use a multi-headed 
attention mechanism and self-attention. Let
be $t_1, \dots, t_n$ be a sequence of input tokens. Then a single 
head will create for each input token $t_i$ an output $y_i$ which is a 
linear combination of the other tokens given as input. It's complexity
is $O(kn^2)$, where k are factors independant of input size. This process is 
repeated for multiple heads with different parameters (learned). Their
outputs are concatened and fed into a fully connected network that combines 
their ouputs. The architecture is complex and impractical to spell out
in greater detail here.

\subsection*{G-flow nets}
*insert*

\section*{Currenctly used classical algorithms (State of The Art)}
There is currently little research and writing on learning 
learning with deep learning algorithms. Instead, biology-specific algorithms 
biology-specific algorithms are favoured, as well as clustering algorithms. 
clustering algorithms. For example, this article from January 2023 uses 
an original algorithm that combines clustering methods to find an optimal 
an optimal aptamer from a selection. 
https://pubs.acs.org/doi/pdf/10.1021/acssynbio.2c00462.
However, some recent papers use deep learning. 
"Machine learning guided aptamer refinement 
and discovery" (https://www.nature.com/articles/s41467-021-22555-9) 
uses a standard MLP neural network to find the most compatible (high affinity) aptamers 
compatible (high affinity) aptamers with target molecules. The estimation of 
free energy is a sub-step of the affinity calculation. It performs a 
truncation step to minimise the length of the aptamer without altering its properties. 
Another deep learning model with aptamers is AptaNet 
(https://www.nature.com/articles/s41598-021-85629-0). This model uses an 
MLP and a CNN to learn the relationship between aptamers and target proteins 
proteins (Aptamere-protein relations or API). The MLP works best, with a 
test accuracy of 91.38%. This neural network performs significantly better than more traditional 
algorithms such as SVM, KNN and random forests. This model 
uses a very detailed database containing numerous auxiliary variables 
measured in the laboratory for each individual, but with only 1000 individuals. 
No published aptamer model uses transformers or RNNs to predict free energy, so the 
predict free energy, so our method would be original in this field.
\bibliography{bibliography}
\bibliographystyle{plain}
\end{document}
