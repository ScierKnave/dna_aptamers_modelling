\documentclass{article}
\title{Review of the litterature}
\author
{
    Guillermo Martinez (matricule x)
    \and
    Dereck Piché (matricule 20177385)
    \and
    Jonas Gabirot (matricule 20185863)
}


\begin{document}
\maketitle

\section*{Aptamers}
Nucleic acids (DNA and RNA) carry the instructions on 
how an organism can grow, develop and replicate. Recent
 developments in our ability to generate large populations of 
 degenerate oligonucleotides and isolate functional nucleic acid 
 molecules that can bind to a specific target have led to the in 
 vitro evolution of nucleic acid molecules for other than biological functions. 
 These affinity reagents based on DNA or RNA are referred to as aptamers 
 and posses two fundamental characteristics: their ability to fold 
 into shapes with defined function (phenotypes) and the ability of 
 their respective sequences (genotypes) to be replicated in vitro to 
 produce progeny molecules with similar characteristics to their 
 parent sequence. Aptamer-engineering consists in exploiting their 
 unique phenotype-genotype connection to amplify individual molecules 
 with desired phenotypes, and thus, optimizing their binding properties 
 to the desired analyte.  For example, through directed evolution 
 procedures such as Systematic Evolution of Ligands by Exponential 
 Enrichment [SELEX], one can apply recursively the principles of 
 natural selection on a large population of nucleic acid sequences 
 incubated with a target— i.e. select and amplify the molecules 
 that bind to the target to create a new population of molecules 
 that are more enriched in sequences that can perform the desired 
 function, until the pool is dominated by the sequences that bind 
 with high affinity to the desired target. These resulting sequences 
 with high affinity towards a specific analyte are called aptasensors 
 (Dunn et al., 2017. Their relatively simple chemical structure 
 (Jeddi et al., 2017)— i.e. sequences composed of roughly of 20 to 
 100 nucleotides in length (Zhang et al., 2019)— is crucial for the 
 insertion of electrochemical or fluorescent reporter molecules, as 
 well as surface-binding agents. When binding with its target, the 
 aptamer undergoes a conformation change which can be exploited to 
 generate an analytical signal (Jeddi et al., 2017). Consequently, 
 aptasensors have been found to be very useful in tracking the propagation 
 of various molecules in their respective environments such as pathogens, 
 toxins, antibiotics and pesticides in food, water and soil samples (Dunn et al., 2017),
  as well as adenosine triphosphate in cells ((Zhang et al., 2019).  
Although DNA aptamers are more stable and more robust than their RNA counterparts, 
the large array of computational tools available for single stranded* 
RNA structure prediction are less pervasive for its DNA counterpart. 
In fact, the computational tools available for DNA were restricted to 
model only double-stranded DNA structures until 2017. Correctly predicting 
the 3-dimensional structure of single-stranded DNA hairpins and other 
more complex structures from 1 dimensional sequences has the potential 
to not only revolutionize aptamer-engineering process but also to bolster
 the range of application for aptasensors in more difficult environments (Jeddi et al., 2017)
 The most crucial difficulty in aptamer-analyte binding analysis is the 
 pre-folding of the aptamer to the correct equilibrium structure. Each DNA
  sequence may adopt a variety of folded structures and brute force 
  techniques such as naïve molecular dynamics search are exceedingly 
  computationally expensive. End-to-End-DNA [E2EDNA]— an end-to-end aptamer-analyte 
  binding pipeline for UTP complexing with simple hairpins—generates a set of 
  structures a given aptamers is likely to adopt, as well as their respective 
  probabilities given molecular dynamics simulations with appropriate force-fields. 
  The pipeline finally implements ‘NUPACK’ and ‘seqfold’ Python packages to identify
   the minimum free energy structure (Kilgour et al., 2021) using nearest-neighbor 
   empirical parameters of a given temperature and ionic strength specified by the
    user (Zadeh et al., 2011). 
The purpose of this research is therefore to train deep learning neural 
networks with randomly generated DNA sequences to predict the minimum 
free energy structure given by ‘NUPACK’.
The most stable aptamer sequences will be potential candidates to 
undergo the entire E2EDNA protocol and be tested on their binding 
affinity to a wide range of analyte of interest. The aptamers that 
are the most stable and possessing the highest binding affinity will 
be potential candidates to be synthetized and used to solve specific 
problems such as trace the oil molecules in the oceans after a spill. 
Given that shorter random-sequence libraries of doubly modified aptamers 
usually posses a higher binding affinity for a target than larger 
traditional sequences (30-mer VS 40-mer random regions), the length 
of our random sequences will be of 30-mer (Dunn et al., 2017). 

\section*{Machine learning algorithms}
\subsection*{Multilayer Perceptrons}
*insert*
\subsection*{Reccurent Neural Networks}
The broad definition of a recurrent neural network 
is that there are some cycles in the layers. 
Typical recurrent network adapted for sequences 
record previous outputs by storing information in the so called
hidden state of the network. Since you have to process each token one at 
a time, their is no parallelization. 
One advantage of recurrent neural network is that they grow 
linearly with respect to the amount of tokens as input, which means 
that their really isn't a practical limit to the number of tokens that
it can process with reasonable time complexity. 
\subsection*{Transformers}
According to our assumptions, the transformer 
architecture \cite{transformers} is by far the most appropriate
for our task. Transformers use a multi-headed 
attention mechanism and self-attention. Let
be $t_1, \dots, t_n$ be a sequence of input tokens. Then a single 
head will create for each input token $t_i$ an output $y_i$ which is a 
linear combination of the other tokens given as input. It's complexity
is $O(kn^2)$, where k are factors independant of input size. This process is 
repeated for multiple heads with different parameters (learned). Their
outputs are concatened and fed into a fully connected network that combines 
their ouputs. The architecture is complex and impractical to spell out
in greater detail here.

\subsection*{G-flow nets}
*insert*

\section*{Currenctly used classical algorithms (State of The Art)}
There is currently little research and writing on learning 
learning with deep learning algorithms. Instead, biology-specific algorithms 
biology-specific algorithms are favoured, as well as clustering algorithms. 
clustering algorithms. For example, this article from January 2023 uses 
an original algorithm that combines clustering methods to find an optimal 
an optimal aptamer from a selection. 
https://pubs.acs.org/doi/pdf/10.1021/acssynbio.2c00462.
However, some recent papers use deep learning. 
"Machine learning guided aptamer refinement 
and discovery" (https://www.nature.com/articles/s41467-021-22555-9) 
uses a standard MLP neural network to find the most compatible (high affinity) aptamers 
compatible (high affinity) aptamers with target molecules. The estimation of 
free energy is a sub-step of the affinity calculation. It performs a 
truncation step to minimise the length of the aptamer without altering its properties. 
Another deep learning model with aptamers is AptaNet 
(https://www.nature.com/articles/s41598-021-85629-0). This model uses an 
MLP and a CNN to learn the relationship between aptamers and target proteins 
proteins (Aptamere-protein relations or API). The MLP works best, with a 
test accuracy of 91.38%. This neural network performs significantly better than more traditional 
algorithms such as SVM, KNN and random forests. This model 
uses a very detailed database containing numerous auxiliary variables 
measured in the laboratory for each individual, but with only 1000 individuals. 
No published aptamer model uses transformers or RNNs to predict free energy, so the 
predict free energy, so our method would be original in this field.
\\\\
1.	Dunn, M., Jimenez, R. and Chaput, J. Analysis of aptamer discovery and technology. Nat Rev Chem 1, 0076 (2017). https://doi.org/10.1038/s41570-017-0076
\\
2.	Jeddi, I., and Saiz, L. (2017). Three-dimensional modeling of single stranded DNA hairpins for aptamer-based biosensors. Scientific reports, 7(1), 1178. https://doi.org/10.1038/s41598-017-01348-5
\\
3.	Kilgour, M., Liu, T., Walker, B. D., Ren, P., and Simine, L. (2021). E2EDNA: Simulation Protocol for DNA Aptamers with Ligands. Journal of chemical information and modeling, 61(9), 4139–4144. https://doi.org/10.1021/acs.jcim.1c00696
\\
4.	Zhang, Y., Lai, B. S., and Juhas, M. (2019). Recent Advances in Aptamer Discovery and Applications. Molecules (Basel, Switzerland), 24(5), 941. https://doi.org/10.3390/molecules24050941 
\\
5.	Zadeh, J. N., Steenberg, C. D., Bois, J. S., Wolfe, B. R., Pierce, M. B., Khan, A. R., Dirks, R. M., and Pierce, N. A. (2011). NUPACK: Analysis and design of nucleic acid systems. Journal of computational chemistry, 32(1), 170–173. https://doi.org/10.1002/jcc.21596 

\bibliography{bibliography}
\bibliographystyle{plain}
\end{document}
