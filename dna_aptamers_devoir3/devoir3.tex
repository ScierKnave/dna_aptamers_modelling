\documentclass{article}
\title{Devoir 3: Revue of the litterature}
\author
{
    Guillermo Martinez (matricule x)
    \and
    Dereck Piché (matricule 20177385)
    \and
    Jonas Gabirot (matricule 20185863)
}


\begin{document}
\maketitle

\section*{Aptamers}
\subsection*{Description of aptamers}
*insert*
\subsection*{Motivation behind the study of Aptamers}
*insert*

\section*{Machine learning algorithms}
\subsection*{Multilayer Perceptrons}
*insert*
\subsection*{Reccurent Neural Networks}
The broad definition of a recurrent neural network 
is that there are some cycles in the layers. 
Typical recurrent network adapted for sequences 
imply the recording of a certain state in the network, 
which is called the hidden state of the reference neural network. 
Most classical recurrent neural networks were not made to be used for 
parallelization, which means that you have to process each token one at 
a time. One advantage of recurrent neural network is that they grow 
linearly with respect to the amount of tokens as input, which means 
that you can input a really, really huge. Um input, and it will still 
work, which is different from transformers, because the complexity 
implies a quadratic term over the number of input tokens, which makes 
it practically impossible for large documents. However, for our 
particular problem, transformers should have no problem at. At processing 
tokens, tokens in the range of 
treaty to 100, which is what is? Actually relevant. And with respect to abdomirrs?
\subsection*{Transformers}
According to our assumptions, the *transformer* architecture is by far the most appropriate
most appropriate for our task. Transformers use a multi-headed 
a multi-headed attention mechanism and self-attention. Let
be a sequence of input tokens. Then the attention mechanism 
mechanism will create for each input token another token which is a 
linear combination of the other given input vectors. Thus, 
we \cite{transformers}. As we assume a strong correlation 
between the input elements, (...) Positional encodings ()

\subsection*{G-flow nets}
*insert*

\section*{Currenctly used classical algorithms (State of The Art)}
There is currently little research and writing on learning 
learning with deep learning algorithms. Instead, biology-specific algorithms 
biology-specific algorithms are favoured, as well as clustering algorithms. 
clustering algorithms. For example, this article from January 2023 uses 
an original algorithm that combines clustering methods to find an optimal 
an optimal aptamer from a selection. 
https://pubs.acs.org/doi/pdf/10.1021/acssynbio.2c00462.
However, some recent papers use deep learning. 
"Machine learning guided aptamer refinement 
and discovery" (https://www.nature.com/articles/s41467-021-22555-9) 
uses a standard MLP neural network to find the most compatible (high affinity) aptamers 
compatible (high affinity) aptamers with target molecules. The estimation of 
free energy is a sub-step of the affinity calculation. It performs a 
truncation step to minimise the length of the aptamer without altering its properties. 
Another deep learning model with aptamers is AptaNet 
(https://www.nature.com/articles/s41598-021-85629-0). This model uses an 
MLP and a CNN to learn the relationship between aptamers and target proteins 
proteins (Aptamere-protein relations or API). The MLP works best, with a 
test accuracy of 91.38%. This neural network performs significantly better than more traditional 
algorithms such as SVM, KNN and random forests. This model 
uses a very detailed database containing numerous auxiliary variables 
measured in the laboratory for each individual, but with only 1000 individuals. 
No published aptamer model uses transformers or RNNs to predict free energy, so the 
predict free energy, so our method would be original in this field.
\bibliography{bibliography}
\bibliographystyle{plain}
\end{document}
