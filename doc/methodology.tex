\documentclass{article}
\usepackage{amsmath}
\title{Methodology}

\author
{
    Dereck Piché \and
    Guillermo Martinez \and
    Jonas Gabirot \and
}

\begin{document}
\maketitle

\section{Task}
Let us briefly resume our task once again in order to make this document
self-contained.

\section{Planned Methodology}
\subsection{Training dataset}
First, we will have to generate our training data. We shall implement a 
simple python script which uses the NUPACK python library in order
to generate a .json file containing our training data. We randomely 
generated a regressively labeled datasets of a milion DNA strands of length $30$.
\subsection{Baseline Algorithms}

\subsubsection{Multilayered Perceptron}

\subsubsection{AdaBoost}
Canceled.

\subsection{Advanced Algorithms}

\subsubsection{Recurrent Neural Network}
Recurrent neural network (RNN) is a deep lRecurrent neural network 
(RNN) is a deep learning architecture used for sequential data prediction
using both current and past inputs. earning architecture used for sequential 
data prediction using both current and past inputs. 

\subsubsection{Tranformer}
\paragraph{Architecture}
The transformer architecture was initially made to translate text.
However, it's subsequent use was mostly tied to token generation. 
This use only required the decoder to be part of the architecture, 
and recursively fed the predicted token in the current input while 
truncating if the input size was over the limit.
Our task is vastly different. We are dealing with a regressive task,
since we are trying to learn a function of the form $R^n \mapsto R$.
Thus, we shall only keep the decoder part of the transformer architecture. 
We only need to set the feedworward neural network in the decoder such that 
it only has one output neuron.
\paragraph{Cost Function}
Since this is a regressive task, we shall use various instances of the 
$l_p$-norm as our cost function.

\paragraph{Tokenisation, encoding, positionnal encoding}
Since the tranformer learns the embedding in the attention heads, \cite{transformers} we
shall simply use an integer mapping for the set of tokens $\{A,C,G,T\}$ as opposed to 
one-hot encoding. This is done partly due to the way the Pytorch library works.

We will train and compare our the accuracy of our decoder-only tranformer with 
respect to the absence and inclusion of positionnal embedding to the input tokens.


\paragraph{Advantages}
What makes the decoder-only transformer different from other models? What are
we taking advantage of by it's use? As opposed to recurrent neural networks, 
this model is less sequential in nature. We are predicting by 
taking the sequence of tokens all at once. We have high hopes for the 
distributivity of attention made possible by the head multiplicity.
We can imagine that there is high importance between the ends of the DNA 
sequence, and at the center. A transformer, given enough data, would 
be able to take advantage of this structure in order to simplify the task.


\subsection{Form of result analysis}

\section{Preliminary Results and Challenges Ahead}

\section{Changes from the initial plan}

\bibliography{bibliography}
\bibliographystyle{plain}
\end{document}